%---------DO NOT EDIT THIS INDENTED SECTION
	% Preamble
	\documentclass[11pt,reqno,oneside,a4paper]{article}
	\usepackage[a4paper,includeheadfoot,left=35mm,right=35mm,top=00mm,bottom=30mm,headheight=40mm]{geometry} %sets up the margins
	\input{../texHead-Proof-Standard} % Use the standard texHead for this module. You should not edit this file.
	\input{../texHead-Proof-Theorems} % Use the standard theorem definitions for this module. You should not edit this file.
	%---The following code defines the title, author, and date of the document.
	\title{Practice proofs \#5}
	\author{Anonymous}
	\date{\today}   % Using \today automatically updates to the document's build date
%----------------------------------
%---------IF YOU WANT TO DEFINE YOUR OWN MACROS, YOU CAN DO SO FROM HERE ...

%---------... TO HERE
\begin{document}
\maketitle
\thispagestyle{fancy}

%-----------EDIT FROM HERE

\begin{abstract}
	\input{PracticeProofsStandardAbstract.tex}
	
	I fitted the whole model solutions onto $1.5$ pages.
\end{abstract}

We study how cardinality interacts with subsets and power sets.

Recall that we say that two sets are equinumerous if they have the same cardinality.
The following proposition says that the operation ``take the power set'' preserves equinumerosity.
	
\begin{prop} \label{prop:PowerSetsPreserveEquinumerosity}
	Suppose $\# X=\# Y$.
	Then $\# \power(X)=\# \power(Y)$.
	Moreover, if $\# X\leq\# Y$, then $\# \power(X)\leq\# \power(Y)$.
\end{prop}

\begin{proof}
We know from Cantor's theorem that the cardinality of the power set of a set of cardinality $n$ equals to $2^n$. Therefore, where $\# X=\# Y = n $, $\# \power(X)= 2^n = \# \power(Y)$. From Cantor's theorem we also know that  if $a = \# X\leq\# Y = b$, then $\# \power(X)\leq\# \power(Y)$, because $2^a \leq 2^b$.
\end{proof}

\begin{prop} \label{prop:SubsetCardinality}
	If $X\subset Y$, then $\# X\leq \# Y$.
\end{prop}

\begin{proof}
If $X\subset Y$, then $\forall a$, $a \in X$, $	\exists! b \in Y$, such that $a=b$. Since each element of $X$ corresponds to (or, more especifically, equals to) a unique element of $Y$, there is an injection from $X$ to $Y$. The domain of an injection is smaller than or equal to its codomain, so  $\# X\leq \# Y$.
\end{proof}

\begin{cor} \label{cor:IntersectionCardinality}
	$\# (X\cap Y) \leq \# X$.
\end{cor}

\begin{proof}
Since all of the elements in the intersection of $X$ and $Y$ are also elements of $X$, $(X \cap Y)$ is a subset of $X$. From proposition 2, we know that the cardinality of a subset of a set is less than or equal to the cardinality of that set. Therefore, $\#(X\cap Y) \leq \#X$.
\end{proof}

\end{document}

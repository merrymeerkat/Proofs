\documentclass[11pt,reqno,oneside,a4paper]{article}

\usepackage{amsmath}
\usepackage{amssymb}
\usepackage{amsthm}

\newtheorem{theorem}{Theorem}
\newtheorem{corollary}{Corollary}[theorem]
\newtheorem{lemma}[theorem]{Lemma}

\title{A bijection from the unit interval to the whole real line}
\author{Daniela}
\date{\today}  


\begin{document}
\maketitle


We explicitly construct a bijection mapping from the unit interval to the whole real line.

\begin{theorem} \label{thm:BijectionRInterval}
	Define $f : (0,1) \to \mathbb{R}$ by
	$$
		f(x) =\frac{1}{1-x} - \frac{1}{x}.
	$$
	The function $f$ is a bijection.
\end{theorem}

\begin{proof}

To prove that $f$ is a bijection, we will first show that it is an injection.

We define a strictly increasing function $g: X \rightarrow Y$. By definition, $\forall a, b \in X$ such that $a > b$, $g(a) > g(b)$. As a result, if $g(a) = g(b)$, it is necessarily the case that $a = b$. Therefore, strictly increasing functions are injective. We will show that $f$ is strictly increasing.

To do so, we will look separately at the two terms that are in $f$, and determine what happens to each of them as $x$ increases.

Firstly, the value of $\frac{1}{1-x}$ increases as $x$ increases within the range of $(0,1)$. This happens because the value of $1 - x$ decreases whenever we start with a positive $x$ and increase the value of $x$. Fractions get larger as their denominators decrease, so $\frac{1}{1-x}$ is strictly increasing for this range.

The opposite happens to $\frac{1}{x}$. As $x$ increases within the range $(0,1)$, the term as a whole decreases, since a greater denominator leads to a smaller number.

Therefore, we know this about the terms of $f(a)$ and $f(b)$, where  $a, b \in (0,1)$ and $a > b$:

$$\frac{1}{1-a} > \frac{1}{1-b} $$ 
and 
$$\frac{1}{a} < \frac{1}{b} $$

Therefore, $\left(\frac{1}{1-a} - \frac{1}{a}\right) >  \left(\frac{1}{1-b} - \frac{1}{b}\right)$, showing that $f(a) > f(b)$ for $a>b$.

Now, let us prove that $f$ is surjective. For this, we will use the intermediate value theorem. This theorem states that, for a function $g$ that is continuous over an interval $[c,d]$, $g$ will take on every value between $f(c)$ and $f(d)$. 

Now, we know that $\frac{1}{1-x}$ is continuous over the range $(0,1)$. That is because the range of $x$ is continuous and there is no $x$ within this range that would not yield a value for $\frac{1}{1-x}$. Similarly, $\frac{1}{x}$ is also continous for this range. Our $f$ is thus a subtraction of two continuous terms, and thus must be continuous as well.

Now, suppose $N$. Then, suppose $x_1 = \frac{1}{N}$ and $x_2 = 1 + \frac{1}{N}$. That means that $lim_{N\to\infty} x_1 = 0$ and $lim_{N\to\infty} x_2 = 1$. We can then say our function $f$ is on the range $[x_1, x_2]$. 

Since $f$ is continuous over $[x_1, x_2]$, it must take every value between $f(x_1)$ and $f(x_2)$. Therefore, $f$ is surjective.

Since $f$ is injective and surjective, it is bijective.

\end{proof}

\end{document}


\documentclass[11pt,reqno,oneside,a4paper]{article}

\usepackage{amsmath}
\usepackage{amssymb}
\usepackage{amsthm}

\newtheorem{theorem}{Theorem}
\newtheorem{lemma}[theorem]{Lemma}
\newtheorem{proposition}[theorem]{Proposition}
\newtheorem{corollary}[theorem]{Corollary}

\title{Cardinalities}
\author{Daniela}
\date{\today}  


\begin{document}
\maketitle

\begin{proposition} \label{proposition:PowerSetsPreserveEquinumerosity}
	Suppose $\# X=\# Y$.
	Then $\# \mathcal{P}(X)=\# \mathcal{P}(Y)$.
	Moreover, if $\# X\leq\# Y$, then $\# \mathcal{P}(X)\leq\# \mathcal{P}(Y)$.
\end{proposition}

\begin{proof}
We know from Cantor's theorem that the cardinality of the power set of a set of cardinality $n$ equals to $2^n$. Therefore, where $\# X=\# Y = n $, $\# \mathcal{P}(X)= 2^n = \# \mathcal{P}(Y)$. From Cantor's theorem we also know that  if $a = \# X\leq\# Y = b$, then $\#\mathcal{P}(X)\leq\# \mathcal{P}(Y)$, because $2^a \leq 2^b$.
\end{proof}

\begin{proposition} \label{prop:SubsetCardinality}
	If $X\subset Y$, then $\# X\leq \# Y$.
\end{proposition}

\begin{proof}
If $X\subset Y$, then $\forall a$, $a \in X$, $	\exists! b \in Y$, such that $a=b$. Since each element of $X$ corresponds to (or, more especifically, equals to) a unique element of $Y$, there is an injection from $X$ to $Y$. The domain of an injection is smaller than or equal to its codomain, so  $\# X\leq \# Y$.
\end{proof}

\begin{corollary} \label{cor:IntersectionCardinality}
	$\# (X\cap Y) \leq \# X$.
\end{corollary}

\begin{proof}
Since all of the elements in the intersection of $X$ and $Y$ are also elements of $X$, $(X \cap Y)$ is a subset of $X$. From proposition 2, we know that the cardinality of a subset of a set is less than or equal to the cardinality of that set. Therefore, $\#(X\cap Y) \leq \#X$.
\end{proof}

\end{document}

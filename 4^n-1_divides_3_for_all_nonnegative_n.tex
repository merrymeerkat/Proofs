\documentclass[11pt,reqno,oneside,a4paper]{article}

\usepackage{amsmath}
\usepackage{amssymb}
\usepackage{amsthm}

\newtheorem{theorem}{Theorem}
\newtheorem{corollary}{Corollary}[theorem]
\newtheorem{lemma}[theorem]{Lemma}

\title{Practice proofs \#2}
\author{Daniela}
\date{\today}  

\begin{document}
\maketitle


\begin{theorem} 
	If $n$ is a nonnegative integer, then $3 \mid (4^n-1)$. \\
\end{theorem}

\begin{proof}

We will prove theorem 1 by induction.

Firstly, we test it for the base case of n = 0. We thus have:

$$3 \mid (4^0-1) $$
$$= 3 \mid (1-1) $$
$$= 3 \mid (0). $$

Since 0 is divisible by three (zero is divisible by any integer apart from itself), the theorem holds for the base case.

Now, assuming that the theorem holds for some $n \in \mathbb{N}$, we need to prove that it holds for $n + 1$ as well. For $n+1$, we have:

$$P(n+1) = 4^{n+1} - 1 .$$

Re-arranging it, we get:

$$4^{n+1} - 1 $$
$$=  4 \times 4^n - 1 $$
$$=  3(4^n) + 4^n - 1. $$

We are assuming that $4^n - 1$ is divisible by 3. Therefore, we can re-write this term $3k$, with $k\in\mathbb{N^0}$. Then, we get

$$P (n+1) =  3(4^n) + 3k $$
$$ = 3(4^n + k) $$

We thus see that if P(n) is divisible by 3, P(n+1) must also be divisible by 3. Theorem 1 holds.

\end{proof}

\end{document}
